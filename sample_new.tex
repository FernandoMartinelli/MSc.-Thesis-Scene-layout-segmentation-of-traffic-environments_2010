%%%%%%%%%%%%%%%%%%%%%%%%%%%%%%%%%%%%%%%%%%%%%%%%%%%%%%%%%%%%%%%%%%%%%%%%%%%%%%%%
%2345678901234567890123456789012345678901234567890123456789012345678901234567890
%        1         2         3         4         5         6         7         8

%\documentclass[letterpaper, 10 pt, conference]{ieeeconf}  % Comment this line out
                                                          % if you need a4paper
\documentclass[a4paper, 10pt, conference]{ieeeconf}      % Use this line for a4
                                                          % paper

\IEEEoverridecommandlockouts                              % This command is only
                                                          % needed if you want to
                                                          % use the \thanks command
\overrideIEEEmargins
% See the \addtolength command later in the file to balance the column lengths
% on the last page of the document



% The following packages can be found on http:\\www.ctan.org
%\usepackage{graphics} % for pdf, bitmapped graphics files
%\usepackage{epsfig} % for postscript graphics files
%\usepackage{mathptmx} % assumes new font selection scheme installed
%\usepackage{times} % assumes new font selection scheme installed
%\usepackage{amsmath} % assumes amsmath package installed
%\usepackage{amssymb}  % assumes amsmath package installed

\title{\LARGE \bf
MSc. Thesis -- Scene layout segmentation of traffic environments
}

%\author{ \parbox{3 in}{\centering Huibert Kwakernaak*
%         \thanks{*Use the $\backslash$thanks command to put information here}\\
%         Faculty of Electrical Engineering, Mathematics and Computer Science\\
%         University of Twente\\
%         7500 AE Enschede, The Netherlands\\
%         {\tt\small h.kwakernaak@autsubmit.com}}
%         \hspace*{ 0.5 in}
%         \parbox{3 in}{ \centering Pradeep Misra**
%         \thanks{**The footnote marks may be inserted manually}\\
%        Department of Electrical Engineering \\
%         Wright State University\\
%         Dayton, OH 45435, USA\\
%         {\tt\small pmisra@cs.wright.edu}}
%}

\author{Fernando Cervigni Martinelli. A Thesis Submitted for the Degree of MSc Erasmus Mundus
in Vision and\\ Robotics (VIBOT), carried out at the Honda Research Institute Europe GmbH% <-this % stops a space
% \thanks{This work was not supported by any organization}% <-this % stops a space
% \thanks{H. Kwakernaak is with Faculty of Electrical Engineering, Mathematics and Computer Science,
%         University of Twente, 7500 AE Enschede, The Netherlands
%         {\tt\small h.kwakernaak@autsubmit.com}}%
% \thanks{P. Misra is with the Department of Electrical Engineering, Wright State University,
%         Dayton, OH 45435, USA
%         {\tt\small pmisra@cs.wright.edu}}%
}


\begin{document}



\maketitle
\thispagestyle{empty}
\pagestyle{empty}


%%%%%%%%%%%%%%%%%%%%%%%%%%%%%%%%%%%%%%%%%%%%%%%%%%%%%%%%%%%%%%%%%%%%%%%%%%%%%%%%
\begin{abstract}

At least 80\% of the traffic accidents in the whole world are caused by human mistakes. Whether drivers are too tired, drunk or speeding, most accidents have their root in the improper behavior of drivers. Many of these accidents would be avoided if cars were equipped with some kind of intelligent system able to detect wrong actions of the driver and autonomously intervene controlling the car. Such an advanced driver assistance system needs to be able to understand the car environment and, from that information, predict the expected behavior of the driver at every instant. As a step towards scene understanding, a system has been implemented that is capable of performing semantic segmentation and classification of road scene video sequences. Some important classes for the prediction of the driver behavior are, for example, `road', `sidewalk', `car', `building' and so on. Our system builds on cutting-edge supervised segmentation and classification techniques, integrating, within a Conditional Random Field model, cues such as color, location, texture and also spatial context between classes. The CamVid database, which contains challenging inner-city road video sequences with very precise ground truth, has been used for assessing the quality of our segmentation and for the comparison with the state of the art.

\end{abstract}


%%%%%%%%%%%%%%%%%%%%%%%%%%%%%%%%%%%%%%%%%%%%%%%%%%%%%%%%%%%%%%%%%%%%%%%%%%%%%%%%
\section{INTRODUCTION}

\subsection{Motivation}

Within the Honda Research Institute Europe (HRI-EU), the Attentive Co-Pilot project (ACP) conducts research on a multi-function Advanced Driver Assistance System (ADAS). It is desired and to be expected that, in the future, cars will autonomously respond to inappropriate actions taken by the driver. If he or she does not stop the car when the traffic lights are red or falls asleep and slowly deviates from the normal driving course, the car should trigger an emergency procedure and somehow warn the driver. It would be even safer if the car had the capability of not only recognizing it and warning the driver when they act imprudently, but also of taking over control and safely correcting the driver's wrong actions.

If, however, this Advanced Driver Assistance System is to become responsible for saving lives, in a critical real-time context, it cannot afford to fail. In order to manage the extremely complicated task of building such an intelligent system, many smaller problems have to be successfully tackled. One of the main ones is related to understanding and adequately representing the environment in which the car operates. For that, a variety of sensors and input data can be used. Indeed, participants of the DARPA Urban Challenge~\cite{darpa:darpa}, which requires autonomous vehicles to drive through specific routes in a city, rely on a wide range of sensors such as GPS, Radar, Lidar, inertial guidance systems as well as on the use of annotated maps.

One of our aspirations, though, is to achieve the task of scene understanding and enviroment representation only by `seeing' the world---like humans do with the greatest of ease---using an off-the-shelf camera mounted in the car. By ruling out the use of complicated equipment and sensing techniques, we aim at, once a reliable driver assistance system is achieved, manufacturing it cheap enough for it to be highly scalable. Considering their great potential of increasing the safety of drivers---and therefore also of pedestrians, bicyclists, and all parties involved in the traffic---, such advanced driver assistance systems will most likely become an indispensable car component, like today's seat-belts.

\subsection{Relevance of segmentation} 

A first step to understanding and representing the world surrounding the car is to segment the images acquired by the camera in meaningful regions and objects. In our case, meaningful regions should be understood as the regions that may have any influence on the behavior of the driver, and, therefore, on the behavior that the car should be able to learn and predict. Examples of such regions are the road, signs, pedestrians, bicyclists, other cars and so on. In order to manage to correctly segment such meaningful reagions, we need to especially consider semantic aspects of the scene rather than only its pixelwise appearance---that is, even if a street is much darker at some points because of shadows, the whole should be segmented as one semantic region. The work described in this paper aims at performing this task of semantic segmentation by exploring the most recent insights of researchers in the field, as well as well-known and state-of-the-art image processing and segmentation techniques. 

% \begin{table}
% \caption{An Example of a Table}
% \label{table_example}
% \begin{center}
% \begin{tabular}{|c||c|}
% \hline
% One & Two\\
% \hline
% Three & Four\\
% \hline
% \end{tabular}
% \end{center}
% \end{table}


%%%%%%%%%%%%%%%%%%%%%%%%%%%%%%%%%%%%%%%%%%%%%%%%%%%%%%%%%%%%%%%%%%%%%%%%%%%%%%%%
\section{PROBLEM DEFINITION}

The main goal of this thesis project is to investigate and implement a system that segments images of road scenes, recognizing its different regions. More specifically, each input color image, $x \in G^{M\times N\times 3}$, where $G = \{ 0, 1, 2, \cdots, 255\}$ and $M$ and $N$ are respectively the image height and width, must be pixelwise segmented. That means that each pixel $i$ on the image has to be assigned one of $N$ pre-defined classes, also called labels, of a set $\mathcal L = \{ l_1, l_2, l_3, \cdots, l_N\}$

As discussed in Chapter~\ref{chap:state_of_the_art}, this task can only be accomplished by supervised training, which means that the system is given labeled training images, from which it should learn in order to subsequently segment new, unseen images. This limitation is mainly due to the fact that unsupervised segmentation techniques do not use ground truth information and cannot \emph{learn} how to semantically segment new images.

Figure~\ref{fig:ideal_segmentation} shows a typical inner-city road scene as considered in this thesis project, as well as its ideal segmentation. The ideal segmentation is taken from the CamVid dataset~\cite{brostow:camvid}. The CamVid database is a recently proposed image database with high-quality, manually-labeled ground truth which we use for our supervised training. The images have been acquired by a car-mounted camera, filming the scene in front of the car while driving in a city. More detail about the CamVid dataset is given in Chapter 5.

\begin{figure}
\centering{
\begin{tabular}{cc}
\includegraphics[height=4cm]{figures/road_scene_example.eps} &
\includegraphics[height=4cm]{figures/road_scene_example_segmented.eps}\\
(a)&(b)
\end{tabular}
}
\caption[Example of ideal semantic segmentation and classification]{(a) An example of a typical inner-city road scene extracted from the CamVid database. (b) The corresponding manually labeled ground truth, taking into account classes like `road', `pedestrian', `sidewalk' and `sky', among others. The goal of the segmentation system to be implemented is to produce, given an image (a), an automatic segmentation that is as close as possible to the ground truth (b).}
\label{fig:ideal_segmentation}
\end{figure}

% \addtolength{\textheight}{-3cm}   % This command serves to balance the column lengths
%                                   % on the last page of the document manually. It shortens
%                                   % the textheight of the last page by a suitable amount.
%                                   % This command does not take effect until the next page
%                                   % so it should come on the page before the last. Make
%                                   % sure that you do not shorten the textheight too much.

%%%%%%%%%%%%%%%%%%%%%%%%%%%%%%%%%%%%%%%%%%%%%%%%%%%%%%%%%%%%%%%%%%%%%%%%%%%%%%%%
\section{ADDITIONAL REQUIREMENTS}

\subsection{Figures and Tables}

Position figures and tables at the tops and bottoms of columns.
Avoid placing them in the middle of columns. Large figures and tables
may span across both columns. Figure captions should be below the figures;
 table captions should be above the tables. Avoid placing figures and tables
  before their first mention in the text. Use the abbreviation ``Fig. 1'',
  even at the beginning of a sentence.
Figure axis labels are often a source of confusion.
Try to use words rather then symbols. As an example write the quantity ``Inductance",
 or ``Inductance L'', not just.
 Put units in parentheses. Do not label axes only with units.
 In the example, write ``Inductance (mH)'', or ``Inductance L (mH)'', not just ``mH''.
 Do not label axes with the ratio of quantities and units.
 For example, write ``Temperature (K)'', not ``Temperature/K''.

\subsection{Numbering}

Number reference citations consecutively in square brackets \cite{c1}.
 The sentence punctuation follows the brackets \cite{c2}.
 Refer simply to the reference number, as in \cite{c3}.
 Do not use ``ref. \cite{c3}'' or ``reference \cite{c3}''.
Number footnotes separately in superscripts\footnote{This is a footnote}
Place the actual footnote at the bottom of the column in which it is cited.
Do not put footnotes in the reference list.
Use letters for table footnotes (see Table I).

\subsection{Abbreviations and Acronyms}

Define abbreviations and acronyms the first time they are used in the text,
even after they have been defined in the abstract. Abbreviations such as
IEEE, SI, CGS, ac, dc, and rms do not have to be defined. Do not use
abbreviations in the title unless they are unavoidable.

\subsection{Equations}

Number equations consecutively with equation numbers in parentheses flush
 with the right margin, as in (1). To make your equations more compact
 you may use the solidus (/), the exp. function, or appropriate exponents.
  Italicize Roman symbols for quantities and variables, but not Greek symbols.
   Use a long dash rather then hyphen for a minus sign. Use parentheses to avoid
    ambiguities in the denominator.
Punctuate equations with commas or periods when they are part of a sentence:
$$\Gamma_2 a^2 + \Gamma_3 a^3 + \Gamma_4 a^4 + ... = \lambda \Lambda(x),$$
where $\lambda$ is an auxiliary parameter.

Be sure that the symbols in your equation have been defined before the
equation appears or immediately following.
Use ``(1),'' not ``Eq. (1)'' or ``Equation (1),''
except at the beginning of a sentence: ``Equation (1) is ...''.

   \begin{figure}[thpb]
      \centering
      %\includegraphics[scale=1.0]{figurefile}
      \caption{Inductance of oscillation winding on amorphous
       magnetic core versus DC bias magnetic field}
      \label{figurelabel}
   \end{figure}

%%%%%%%%%%%%%%%%%%%%%%%%%%%%%%%%%%%%%%%%%%%%%%%%%%%%%%%%%%%%%%%%%%%%%%%%%%%%%%%%
\section{CONCLUSIONS AND FUTURE WORKS}

\subsection{Conclusions}

This is a repeat.
Position figures and tables at the tops and bottoms of columns.
Avoid placing them in the middle of columns. Large figures and tables
may span across both columns. Figure captions should be below the figures;
 table captions should be above the tables. Avoid placing figures and tables
  before their first mention in the text. Use the abbreviation ``Fig. 1'',
  even at the beginning of a sentence.
Figure axis labels are often a source of confusion.
Try to use words rather then symbols. As an example write the quantity ``Inductance",
 or ``Inductance L'', not just.
 Put units in parentheses. Do not label axes only with units.
 In the example, write ``Inductance (mH)'', or ``Inductance L (mH)'', not just ``mH''.
 Do not label axes with the ratio of quantities and units.
 For example, write ``Temperature (K)'', not ``Temperature/K''.


\subsection{Future Works}

This is a repeat.
Position figures and tables at the tops and bottoms of columns.
Avoid placing them in the middle of columns. Large figures and tables
may span across both columns. Figure captions should be below the figures;
 table captions should be above the tables. Avoid placing figures and tables
  before their first mention in the text. Use the abbreviation ``Fig. 1'',
  even at the beginning of a sentence.
Figure axis labels are often a source of confusion.
Try to use words rather then symbols. As an example write the quantity ``Inductance",
 or ``Inductance L'', not just.
 Put units in parentheses. Do not label axes only with units.
 In the example, write ``Inductance (mH)'', or ``Inductance L (mH)'', not just ``mH''.
 Do not label axes with the ratio of quantities and units.
 For example, write ``Temperature (K)'', not ``Temperature/K''.

%%%%%%%%%%%%%%%%%%%%%%%%%%%%%%%%%%%%%%%%%%%%%%%%%%%%%%%%%%%%%%%%%%%%%%%%%%%%%%%%
\section{ACKNOWLEDGMENTS}

The authors gratefully acknowledge the contribution of National Research Organization and reviewers' comments.


%%%%%%%%%%%%%%%%%%%%%%%%%%%%%%%%%%%%%%%%%%%%%%%%%%%%%%%%%%%%%%%%%%%%%%%%%%%%%%%%

References are important to the reader; therefore, each citation must be complete and correct. If at all possible, references should be commonly available publications.

\begin{thebibliography}{99}

\bibitem{c1}
J.G.F. Francis, The QR Transformation I, {\it Comput. J.}, vol. 4, 1961, pp 265-271.

\bibitem{c2}
H. Kwakernaak and R. Sivan, {\it Modern Signals and Systems}, Prentice Hall, Englewood Cliffs, NJ; 1991.

\bibitem{c3}
D. Boley and R. Maier, "A Parallel QR Algorithm for the Non-Symmetric Eigenvalue Algorithm", {\it in Third SIAM Conference on Applied Linear Algebra}, Madison, WI, 1988, pp. A20.

\end{thebibliography}

\end{document}
